\documentclass[a4paper, 10pt, twocolumn]{scrartcl}

% Generale page style.
\usepackage[margin = 0.86cm]{geometry}

% Core packages.
\usepackage[ngerman]{babel}
\usepackage[utf8]{inputenc}
\usepackage[T1]{fontenc}
% Other packages.
\usepackage{amssymb}
\usepackage[german, onelanguage]{algorithm2e}
\usepackage{calc}
\usepackage{csquotes}
\usepackage{enumitem}
\usepackage{float}
\usepackage{kbordermatrix}
\usepackage{listings}
\usepackage{makecell}
\usepackage{mathtools}
\usepackage{multicol}
\usepackage{newunicodechar}
\usepackage{stmaryrd}
\usepackage{tabto}
%\usepackage[disable]{todonotes}
\usepackage{todonotes}
% TikZ.
\usepackage{tikz}
\usetikzlibrary{angles, arrows.meta, backgrounds, calc, positioning, shapes}

% Basic information.
\title{Lösung von Taktfahrplanoptimierungsproblemen durch Modulo-Simplex-Berechnungen}
\author{Fabian Damken}
\date{\today}

% Description-list styling.
\SetLabelAlign{parright}{\parbox[t]{\labelwidth}{\raggedleft#1}}
\setlist[description]{style = multiline, leftmargin = 4cm, align = parright}

\MakeOuterQuote{"}

\tikzset{> = { Latex[length = 2.5mm] }}
\tikzstyle{every path} = [ very thick ]

% New commands.
\newcommand{\arr}{\mathit{arr}}
\newcommand{\const}{\ensuremath{\textrm{const}}}
\newcommand{\C}{\ensuremath{\mathbb{C}}}
\newcommand{\dep}{\mathit{dep}}
\newcommand{\dif}[1]{\ensuremath{\,\mathrm{d}#1}}
\newcommand{\N}{\ensuremath{\mathbb{N}}}
\newcommand{\R}{\ensuremath{\mathbb{R}}}
\newcommand{\Z}{\ensuremath{\mathbb{Z}}}
% Matrix and vector notation.
\newcommand{\mat}[1]{\boldsymbol{#1}}
\renewcommand{\vec}[1]{\boldsymbol{#1}}
\makeatletter
\newcommand{\BIG}{\bBigg@{4}}
\newcommand{\BIGG}{\bBigg@{5}}
\makeatother

\let\oldvdots=\vdots
\renewcommand{\vdots}{\raisebox{2pt}{\(\oldvdots\)}}
\let\oldddots=\ddots
\renewcommand{\ddots}{\raisebox{2pt}{\(\oldddots\)}}

\renewcommand{\kbldelim}{[}
\renewcommand{\kbrdelim}{]}

\newcommand{\bzw}{bzw.~}
\newcommand{\zB}{z.\,B.~}
\renewcommand{\dh}{d.\,h.~}

\newunicodechar{−}{--}

\begin{document}
	\bibliographystyle{alpha}

	\makeatletter
	\begin{center}
		\textbf{\Large Lösung periodischer \mbox{Fahrplanoptimierungsprobleme} \\ \vspace{2mm} durch Modulo-Simplex-Berechnungen} \\ \vspace{2mm}
		\textbf{\large Zusammenfassung von Nachtigall} \\ \vspace{3mm}
		\textbf{\large Fabian Damken} \\ \vspace{1mm}
		{\large \@date}
	\end{center}
	\makeatother
	
	\subsection*{Einleitung und Ziel}
		Das Ziel der Taktfahrplanoptimierung ist die Minimierung unerwünschter Slack-Zeiten (Pufferzeiten). Dies sind beispielsweise die Wartezeiten von Passagieren an Stationen oder Wartezeiten auf der Strecke.
	% end

	\subsection*{Definitionen}
		\subparagraph{Schienennetz}
		Ein \emph{Schienennetz} ist ein System von \emph{Linien} \(\mathcal{L}\) und \emph{Stationen} \(\mathcal{S}\).
	
		\subparagraph{Ereignis}
		Bedient Linie \( L \in \mathcal{L} \) die Station \( S \in \mathcal{S} \), so werden zwei \emph{Ereignisse} definiert. Sei \( (L, \arr, S) \) die Ankunft an Station \(S\) und \( (L, \dep, S) \) die Abfahrt von Station \(S\). Eine Linie entspricht dann einer alternierenden Sequenz von Ankunfts- und Abfahrt-Ereignissen.
	
		\subparagraph{Fahrplan}
		Ein \emph{Fahrplan} \( \vec{\pi} = (\pi_i) \) ordnet jedem Ereignis \( i = (L, \arr, S) \), \bzw \( i = (L, \dep, S) \), einen Zeitpunkt \( \pi_i \in \R \) zu. In einem Taktfahrplan findet dieses Ereignis zu allen Zeitpunkten \( \pi_i + z T \) (\( z \in \Z \)) statt, wobei \( T \) die Taktzeit ("Periode") bezeichnet.
		
		\subparagraph{Vorgang und Spannung}
		Ein \emph{Vorgang} \( a : i \to j \) beschreibt den Übergang von \(i\) zu \(j\). Die Dauer dieses Vorgangs ist die sogenannte \emph{Spannung} \( x_a \coloneqq \pi_j - \pi_i \). Sei \(\mathcal{A}\) die Menge aller Vorgänge.
		
		\subparagraph{Zeiteinschränkungen}
		Jedem Vorgang \( a \in \mathcal{A} \) wird eine \emph{zulässige Dauer} \( \Delta_a = [l_a, u_a] \) zugeordnet, in der sich die Spannung befinden muss, damit der Fahrplan gültig ist. Ein klassischer Fahrplan \(\vec{\pi}\) heißt \emph{zulässig} wenn gilt:
		\begin{equation*}
			\forall (a : i \to j) \in \mathcal{A} : l_a \leq \pi_j - \pi_i \leq u_a
		\end{equation*}
		Ein Taktfahrplan \( \vec{\pi} \) heißt zulässig, wenn gilt:
		\begin{equation*}
			\forall (a : i \to j) \in \mathcal{A} : \exists z_a \in \Z : l_a \leq \pi_j - \pi_i - z_a T \leq u_a
		\end{equation*}
		
		\subparagraph{Ereignisnetzwerk}
		Die Ereignisse und Einschränkungen \( \mathcal{A} \) formen einen Graphen, das sogenannte \emph{Ereignisnetzwerk}.
		
		\subparagraph{Slack-Zeit}
		Mit dem Modulo-Operator\footnote{Sei \( [t]_T \coloneqq \min \{\, t + zT \,\vert\, t + zT \leq,\quad z \in \Z,\quad T = \const \,\} \).} sind die Slack-Zeiten (die Zeit, um die die Spannung geändert werden darf, ohne ungültig zu werden) definiert durch \( y_a^\mathit{low} \coloneqq [x_a - l_a]_T \) (obere Slack-Zeit) und \( y_a^\mathit{upp} \coloneqq [u_a - x_a]_T \) (untere Slack-Zeit).
		
	\subsection*{Das Optimierungsproblem}
		Das Optimierungsproblem ist nun, die gewichteten Slack-Zeiten zu minimieren. Dafür wird jeder Kante \( a \) ein Gewicht \( \omega_a \) zugewiesen, wodurch sich das Optimierungsproblem wie folgt formulieren lässt:
		{\small
		\begin{equation*}
			\min \Bigg\{\, \sum_{a \in \mathcal{A}} \omega_a (x_a - l_a - z_a T) \,\bigg\vert\, \forall a \in \mathcal{A} : l_a \leq x_a - z_a T \leq u_a,\, z_a \in \Z \,\Bigg\}
		\end{equation*}}
		Dies stellt ein Gemischt-Ganzzahliges Programm dar, dessen Lösung NP-schwer ist.
		
		Mit der Netzwerkmatrix ("Edge-Cycle-Matrix") \( \mat{\Gamma} \) und \( \vec{\delta} = \vec{u} - \vec{l} \) sowie \( \vec{b} \coloneqq [-\mat{\Gamma} \vec{l}]_T \), können die zulässigen Slack-Zeiten wie folgt gefasst werden:
		\begin{equation*}
			\mathcal{Y} \coloneqq \big\{\, \vec{y} \in \Z^m \,\vert\, \mat{\Gamma} \vec{x} \equiv_T \vec{b},\quad \vec{0} \leq \vec{y} \leq \vec{\delta} \,\big\}
		\end{equation*}
		Dadurch kann das Optimierungsproblem zu \( \min \big\{\, \vec{\omega}^T \vec{y} \,\vert\, \vec{y} \in \mathcal{Y} \,\big\} \) umgeformt werden. Dies ist das sogenannte \emph{Slack-Modell}.
		
		\paragraph{Dualität zum Minimum-Cost-Flow Problem}
			Werden die Modulo-Parameter \( \vec{z} \) fixiert, kann das Optimierungsproblem mit \( \vec{l}' \coloneqq \vec{l} + \vec{z}T \) und \( \vec{u}' \coloneqq \vec{u} + \vec{z}T \) zur dualen Formulierung des Minimum-Cost-Flow Problems umgeformt werden (es gilt \( \vec{x} = \mat{\Theta} \vec{\pi} \)):
			\begin{equation*}
				\min \big\{\, \vec{\omega}^T (\mat{\Theta}^T \vec{\pi} - \vec{l}') \,\vert\, \vec{l}' \leq \mat{\Theta}^T \vec{\pi} \leq \vec{u}' \,\big\}
			\end{equation*}
			
			Es kann gezeigt werden, dass jeder Extrempunkt dieses Problems einer Spannbaumstruktur \( \mathcal{T} = \mathcal{T}^l + \mathcal{T}^u \) entspricht, wobei die Spannung von jeder Kante in \( \mathcal{T}^l \) auf die untere und in \( \mathcal{T}^u \) auf die obere Schranke festgesetzt wird. Diese Spannbaumstrukturen generieren dann ein eindeutiges \emph{Potential}  \(\vec{\pi}\), welches einem Fahrplan entspricht.
		% end
	% end
	
	\subsection*{Das Modulo-Simplex-Verfahren}
		Das Modulo-Simplex-Verfahren erkundet die Extrempunkte dieses Optimierungsproblems und sucht optimale Spannbäume, wobei die in Ko-Baum-Kanten und Baum-Kanten separierte Netzwerkmatrix \( \mat{\Gamma} = [\mat{N}_\mathcal{T}, \mat{I}_\mathcal{T}^\mathit{co}] \) als Koeffizientenmatrix fungiert, \dh es wird die Kostenfunktion \( \vec{\omega}^T \vec{y} \) unter Beibehaltung der Gleichung \( \mat{\Gamma} \vec{y} \equiv_T \vec{b} \) optimiert (minimiert).
		
		Dazu werden sukzessive Ko-Baum-Kanten gegen Baum-Kanten getauscht, sofern die Kosten dadurch verbessert werden. Dabei berechnen sich die aktualisierten Kosten durch den Austausch von Ko-Baum-Kante \( a_i \) gegen Baum-Kante \( a_j \) durch
		\begin{equation*}
			\tilde{\omega}_{ij} = \omega_j \bigg[ \frac{b_i}{\gamma_{ij}} \bigg]_T \! + \sum_{\substack{k = r \\ k \neq i}}^{m} \omega_k \bigg[ b_k - \frac{\gamma_{kj}}{\gamma_{ij}} b_i \bigg]_T
		\end{equation*}
		das heißt die Änderung \( \Delta\omega_{ij} = \tilde{\omega}_{ij} - \omega \) ist gegeben durch
		\begin{equation*}
			\Delta\omega_{ij} = \omega_j \bigg[ \frac{b_i}{\gamma_{ij}} \bigg]_T \! - \omega_i b_i + \sum_{\substack{k = r \\ k \neq i}}^{m} \omega_k \Bigg( \bigg[ b_k - \frac{\gamma_{kj}}{\gamma_{ij}} b_i \bigg]_T - b_k \Bigg)
		\end{equation*}
	% end
	
	\subsection*{Knotenlokale Verbesserungen}
		Diese sukzessive Verbesserungen führen in den meisten Fällen nur zu einem lokalen Optimum, \dh das globale Optimum wird nie erreicht. Um das Finden eines globalen Optimums zu ermöglichen, wird die Struktur ausgenutzt, dass jede Spannung \(\vec{x}\) (\dh es gilt \( \mat{\Gamma} \vec{x} \equiv_T \vec{0} \)) mit \( \vec{y}' \coloneqq [\vec{y} + \vec{x}]_T \) eine neue Lösung definiert, welche gültig ist, wenn \( \vec{y}' \leq \vec{\delta} \) gilt. Verbessert sich dadurch die Zielfunktion \( \vec{\omega}^T \vec{y} \) (\bzw \( \vec{\omega}^T \vec{y}' \)), so ist \(\vec{y}'\) eine bessere Lösung als \(\vec{y}\).
		
		Ein einzelner Knoten \(i\) induziert nun einen Schnitt \( \vec{\eta}^{(i)} \) mit \( P = \{\, i \,\} \), \dh jede Kante um den Knoten wird geschnitten. Die Addition von \( \delta \) auf das Potential von \(i\) (\( \pi_i' \coloneqq \pi_i + \delta \)) entspricht einer \(\delta\)-fachen Anwendung des Schnitts \( \vec{\eta}^{(i)} \) auf die Slack-Zeiten \( \vec{y} \). Dadurch wird mit \( \vec{y}' \coloneqq \vec{y} + \delta\vec{\eta}^{(i)} \) eine neue Lösung durch den Schnitt \( \vec{\eta}^{(i)} \) bestimmt. Dies wird \emph{Knotenlokale Verbesserung} genannt.
		
		Nach Anwendung eines solchen Schnitts wird das Modulo-Simplex-Verfahren erneut angewendet, um die gefundene modifizierte Lösung weiter zu optimieren. Dadurch ist es möglich, ein besseres Optimum zu finden also nur durch Anwendung des Modulo-Simplex-Verfahrens. Allerdings stellen die Knotenlokalen Verbesserungen nur eine Heuristik dar, \dh das Finden eines globale Optimums ist weiterhin nicht garantiert.
	% end
\end{document}
