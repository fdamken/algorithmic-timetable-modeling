\documentclass[a4paper, 10pt, twocolumn]{scrartcl}

% Generale page style.
\usepackage[margin = 1cm]{geometry}

% Core packages.
\usepackage[ngerman]{babel}
\usepackage[utf8]{inputenc}
\usepackage[T1]{fontenc}
% Other packages.
\usepackage{amssymb}
\usepackage[german, onelanguage]{algorithm2e}
\usepackage{calc}
\usepackage{csquotes}
\usepackage{enumitem}
\usepackage{float}
\usepackage{kbordermatrix}
\usepackage{listings}
\usepackage{makecell}
\usepackage{mathtools}
\usepackage{multicol}
\usepackage{newunicodechar}
\usepackage{stmaryrd}
\usepackage{tabto}
%\usepackage[disable]{todonotes}
\usepackage{todonotes}
% TikZ.
\usepackage{tikz}
\usetikzlibrary{angles, arrows.meta, backgrounds, calc, positioning, shapes}

% Basic information.
\title{Lösung von Taktfahrplanoptimierungsproblemen durch Modulo-Simplex-Berechnungen}
\author{Fabian Damken}
\date{\today}

% Description-list styling.
\SetLabelAlign{parright}{\parbox[t]{\labelwidth}{\raggedleft#1}}
\setlist[description]{style = multiline, leftmargin = 4cm, align = parright}

\MakeOuterQuote{"}

\tikzset{> = { Latex[length = 2.5mm] }}
\tikzstyle{every path} = [ very thick ]

% New commands.
\newcommand{\arr}{\mathit{arr}}
\newcommand{\const}{\ensuremath{\textrm{const}}}
\newcommand{\C}{\ensuremath{\mathbb{C}}}
\newcommand{\dep}{\mathit{dep}}
\newcommand{\dif}[1]{\ensuremath{\,\mathrm{d}#1}}
\newcommand{\N}{\ensuremath{\mathbb{N}}}
\newcommand{\R}{\ensuremath{\mathbb{R}}}
\newcommand{\Z}{\ensuremath{\mathbb{Z}}}
% Matrix and vector notation.
\newcommand{\mat}[1]{\boldsymbol{#1}}
\renewcommand{\vec}[1]{\boldsymbol{#1}}
\makeatletter
\newcommand{\BIG}{\bBigg@{4}}
\newcommand{\BIGG}{\bBigg@{5}}
\makeatother

\let\oldvdots=\vdots
\renewcommand{\vdots}{\raisebox{2pt}{\(\oldvdots\)}}
\let\oldddots=\ddots
\renewcommand{\ddots}{\raisebox{2pt}{\(\oldddots\)}}

\renewcommand{\kbldelim}{[}
\renewcommand{\kbrdelim}{]}

\newcommand{\bzw}{bzw.~}
\newcommand{\zB}{z.\,B.~}
\renewcommand{\dh}{d.\,h.~}

\newunicodechar{−}{--}

\begin{document}
	\bibliographystyle{alpha}

	\makeatletter
	\begin{center}
		\textbf{\Large Lösung periodischer \mbox{Fahrplanoptimierungsprobleme} \\ \vspace{2mm} durch Modulo-Simplex-Berechnungen} \\ \vspace{2mm}
		\textbf{\large Zusammenfassung von Nachtigall} \\ \vspace{3mm}
		\textbf{\large Fabian Damken} \\ \vspace{1mm}
		{\large \@date}
	\end{center}
	\makeatother
	
	\subsection*{Einführung}
		% TODO: Slide 14
	% end

	\subsection*{Definitionen}
		\subparagraph{Schienennetz}
		Ein \emph{Schienennetz} ist ein System von \emph{Linien} \(\mathcal{L}\) und \emph{Stationen} \(\mathcal{S}\).
	
		\subparagraph{Ereignis}
		Bedient Linie \( L \in \mathcal{L} \) die Station \( S \in \mathcal{S} \), so werden zwei \emph{Ereignisse} definiert. Sei \( (L, \arr, S) \) die Ankunft an Station \(S\) und \( (L, \dep, S) \) die Abfahrt von Station \(S\). Eine Linie entspricht dann einer alternierenden Sequenz von Ankunfts- und Abfahrt-Ereignissen.
	
		\subparagraph{Fahrplan}
		Ein \emph{Fahrplan} \( \vec{\pi} = (\pi_i) \) ordnet jedem Ereignis \( i = (L, \arr, S) \), \bzw \( i = (L, \dep, S) \), einen Zeitpunkt \( \pi_i \in \R \) zu. In einem Taktfahrplan findet dieses Ereignis zu allen Zeitpunkten \( \pi_i + z T \) (\( z \in \Z \)) statt, wobei \( T \) die Taktzeit ("Periode") bezeichnet.
		
		\subparagraph{Vorgang und Spannung}
		Ein \emph{Vorgang} \( a : i \to j \) beschreibt den Übergang von \(i\) zu \(j\). Die Dauer dieses Vorgangs ist die sogenannte \emph{Spannung} \( x_a \coloneqq \pi_j - \pi_i \). Sei \(\mathcal{A}\) die Menge aller Vorgänge.
		
		\subparagraph{Zeiteinschränkungen}
		Jedem Vorgang \( a \in \mathcal{A} \) wird eine \emph{zulässige Dauer} \( \Delta_a = [l_a, u_a] \) zugeordnet, in der sich die Spannung befinden muss, damit der Fahrplan gültig ist. Ein klassischer Fahrplan \(\vec{\pi}\) heißt \emph{zulässig} wenn gilt:
		\begin{equation*}
			\forall (a : i \to j) \in \mathcal{A} : l_a \leq \pi_j - \pi_i \leq u_a
		\end{equation*}
		Ein Taktfahrplan \( \vec{\pi} \) heißt zulässig, wenn gilt:
		\begin{equation*}
			\forall (a : i \to j) \in \mathcal{A} : \exists z_a \in \Z : l_a \leq \pi_j - \pi_i - z_a T \leq u_a
		\end{equation*}
		
		\subparagraph{Ereignisnetzwerk}
		Die Ereignisse und Einschränkungen \( \mathcal{A} \) formen einen Graphen, das sogenannte \emph{Ereignisnetzwerk}.
		
		\subparagraph{Slack-Zeit}
		Die dem Modulo-Operator\footnote{Sei \( [t]_T \coloneqq \min \{\, t + zT \,\vert\, t + zT \leq,\quad z \in \Z,\quad T = \const \,\} \).} sind die Slack-Zeiten (die Zeit, um die die Spannung geändert werden darf, ohne ungültig zu werden) definiert durch \( y_a^\mathit{low} \coloneqq [x_a - l_a]_T \) (obere Slack-Zeit) und \( y_a^\mathit{upp} \coloneqq [u_a - x_a]_T \) (untere Slack-Zeit).
		
		\subparagraph{Das Optimierungsproblem}
		Das Optimierungsproblem ist nun, die gewichteten Slack-Zeiten zu minimieren. Dafür wird jeder Kante \( a \) ein Gewicht \( \omega_a \) zugewiesen, wodurch sich das Optimierungsproblem wie folgt formulieren lässt:
		\begin{equation*}
			\min \Bigg\{\, \sum_{a \in \mathcal{A}} \omega_a (x_a - l_a - z_a T) \,\bigg\vert\, \forall a \in \mathcal{A} : l_a \leq x_a - z_a T \leq u_a, \quad z_a \in \Z \,\Bigg\}
		\end{equation*}
		Dies stellt ein Gemischt-Ganzzahliges Programm dar, dessen Lösung NP-schwer ist.
		
		Mit der Netzwerkmatrix ("Edge-Cycle-Matrix") \( \mat{\Gamma} \) und \( \vec{\delta} = \vec{u} - \vec{l} \) sowie \( \vec{b} \coloneqq [-\mat{\Gamma} \vec{l}]_T \), können die zulässigen Slack-Zeiten wie folgt zusammengefasst werden:
		\begin{equation*}
			\mathcal{Y} \coloneqq \big\{\, \vec{y} \in \Z^m \,\vert\, \mat{\Gamma} \vec{x} \equiv_T \vec{b},\quad \vec{0} \leq \vec{y} \leq \vec{\delta} \,\big\}
		\end{equation*}
		Dadurch kann das Optimierungsproblem zu \( \min \big\{\, \vec{\omega}^T \vec{y} \,\vert\, \vec{y} \in \mathcal{Y} \,\big\} \) umgeformt werden. Dies ist das sogenannte \emph{Slack-Modell}.
	% end
\end{document}



































