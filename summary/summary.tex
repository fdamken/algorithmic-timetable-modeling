\documentclass[a4paper, 11pt, twocolumn]{scrartcl}

% Generale page style.
\usepackage[margin = 1cm]{geometry}

% Core packages.
\usepackage[ngerman]{babel}
\usepackage[utf8]{inputenc}
\usepackage[T1]{fontenc}
% Other packages.
\usepackage{amssymb}
\usepackage{csquotes}
\usepackage{enumitem}
\usepackage[mathcal]{euscript}  % Get readable mathcal font.
\usepackage{float}
\usepackage{listings}
\usepackage{makecell}
\usepackage{mathtools}
\usepackage{newunicodechar}
\usepackage{stmaryrd}
\usepackage{tabto}
%\usepackage[disable]{todonotes}
\usepackage{todonotes}

% Basic information.
\title{Lösung periodischer Fahrplanoptimierungsprobleme durch Modulo-Simplex-Berechnungen}
\author{Fabian Damken}
\date{\today}

% Description-list styling.
\SetLabelAlign{parright}{\parbox[t]{\labelwidth}{\raggedleft#1}}
\setlist[description]{style = multiline, leftmargin = 4cm, align = parright}

\MakeOuterQuote{"}

% New commands.
\newcommand{\const}{\ensuremath{\textrm{const}}}
\newcommand{\C}{\ensuremath{\mathbb{C}}}
\newcommand{\dif}[1]{\ensuremath{\,\mathrm{d}#1}}
\newcommand{\N}{\ensuremath{\mathbb{N}}}
\newcommand{\qed}{\hfill \(\Box\)}
\newcommand{\R}{\ensuremath{\mathbb{R}}}
\newcommand{\Z}{\ensuremath{\mathbb{Z}}}
% Matrix and vector notation. Use both boldsymbol and mathbf as the first works
% only for greek letters while the latter works only for latin letters.
\newcommand{\mat}[1]{\boldsymbol{\mathbf{#1}}}
\renewcommand{\vec}[1]{\boldsymbol{\mathbf{#1}}}
\makeatletter
\newcommand{\BIG}{\bBigg@{4}}
\newcommand{\BIGG}{\bBigg@{5}}
\makeatother
\DeclareMathOperator{\arr}{arr}
\DeclareMathOperator{\dep}{dep}

\newcommand{\zB}{z.\,B.~}

\newunicodechar{−}{--}

\begin{document}
	\bibliographystyle{alpha}

	\makeatletter
	\begin{center}
		\textbf{\Large Lösung periodischer \mbox{Fahrplanoptimierungsprobleme} \\ \vspace{2mm} durch Modulo-Simplex-Berechnungen} \\ \vspace{2mm}
		\textbf{\large Zusammenfassung von \cite{nachtigallSolvingPeriodicTimetable2008}} \\ \vspace{3mm}
		\textbf{\large Fabian Damken} \\ \vspace{1mm}
		{\large \@date}
	\end{center}
	\makeatother

	\section{Einleitung}
		In den letzten Jahren sind periodische Fahrpläne und die Optimierung dieser weit in den Fokus der Optimierungsforschung gerückt. Die meisten Ergebnisse bauen auf den Modellen \cite{serafiniMathematicalModelPeriodic1989} auf. Diese Modelle erlauben eine flexible Modellierung von Taktfahrplänen und vielen Voraussetzungen (\zB Vorfahrtbeschränkungen und Grenzwerte für gleichzeitig fahrende Züge).
		
		Die folgenden Abschnitte beschäftigen sich mit der grundlegenden Modellierung solcher Schienennetze.
		
		Ein \emph{Schienennetz} ist ein System an Zuglinien \(\mathcal{L}\) und Stationen \(\mathcal{S}\). Bedient eine Linie \(L \in \mathcal{L}\) eine Station \(S \in \mathcal{S}\), dann sind \( (L, \arr, S) \) und \( (L, \dep, S) \) die Ankunfts- und Abfahrt-Ereignissen von \(L\) an \(S\). Ein periodischer Fahrplan \( \vec{\pi} = (\pi_i) \) weißt jedem Ereignis \( i = (L, \arr, S) \) (oder \( i = (L, \dep, S) \)) einen Zeitpunkt \( \pi_i \in \R \) zu, wobei das Ereignis immer zu den Zeitpunkten \( \pi_i + z_i T \) mit dem Modulo-Parameter \( z_i \in \Z \) stattfindet\footnote{Zur Einfachheit wird angenommen, dass die Zeitperiode \(T\) für das ganze System gleich ist.}. Eine Aktivität \( a : i \to j \) bezeichnet den Prozess, von dem Ereignis \(i\) zu dem Ereignis \(j\) zu gelangen. Eine Aktivität benötigt die Zeit (\emph{Tension}) \( x_a = \pi_j - \pi_i \). Die Menge aller Aktivitäten wir mit \( \mathcal{A} \) bezeichnet.
		
		Jeder Aktivität wird eine Zeitspanne \( \Delta_a = [l_a, u_a] \) zugeordnet, wobei \( l_u \) die minimale und \( u_a \) die maximale Zeit darstellt (obere und untere Schranken). Ein periodischer Fahrplan \( \vec{\pi} \) ist \emph{durchführbar}, wenn \( \forall a \in \mathcal{A} : \exists z_a \in \Z : l_a \leq x_a - z_a T \leq u_a \) gilt. Abgesehen von Start- und Stoppaktivitäten können viele reale Einschränkungen durch solche Zeitspannen definiert werden (\zB Vorfahrtbeschränkungen zur Betriebssicherheit oder Umsteigezeiten für Passagiere). Insgesamt ergibt sich aus den Aktivitäten \(\mathcal{A}\) und den Stationen \(\mathcal{S}\) ein gerichtetes Netzwerk, welches \emph{Ereignis-Aktivitäts-Netzwerk} genannt wird.
		
		Mit dem Modulo-Operator \( [t]_T \coloneqq \min \big\{ t + zT \,\vert\, z + zT \geq 0 \big\} \) wird die obere (\( y_a^\text{low} \coloneqq [x_a - l_a]_T \)) untere (\( y_a^\text{upp} \coloneqq [u_a - x_a]_T \)) \emph{Pufferzeit} ("slack time"). Diese entspricht der Zeit, um welche die Tension \(x_a\) verringert, bzw. erhöht, werden kann, ohne die Einschränkungen zu verletzen.
		
		%Da die unteren und oberen Schranken durch Invertieren der Richtung der Kante \(a\) ausgetauscht werden können, kann 5das Optimierungsproblem allein in Bezug auf die untere Pufferzeit \( y_a^\text{low} \) formuliert werden:
		%\begin{equation*}
		%	\min \Bigg\{ \sum_{a : i \to j} \vec{\omega}_a (x_a - l_a - z_a T) \,\big\vert\, \forall a : i \to j : l_a \leq x_a %- z_a T \leq u_a,\, z_a \in \Z \Bigg \}
		%\end{equation*}
	% end
	
	\section{Mathematische Basis} \todo{Bessere Überschrift ausdenken}
		Mit der Inzidenzmatrix \( \mat{\Theta} = (\theta_{ai}) \) des Netzwerks lässt sich die Tension \( \vec{x} = (x_a) \) ausdrücken durch \( \mat{\Theta}^T \vec{\pi} = \vec{x} \).
		
		Wird zu einem Spannbaum \(\mathcal{T}\) des Netzwerkes eine Kante des Komplementbaums hinzugefügt, ergibt sich ein eindeutig bestimmter Zyklus \(c\). Die Netzwerkmatrix \( \mat{\Gamma} = (\gamma_{ca}) \) enthält eine Zeile pro Kante des Komplementbaums (bzw. pro Zyklus) und eine Spalte pro Kante des Netzwerkes. Die Elemente sind bestimmt durch:
		\begin{equation*}
			\gamma_{ca} \coloneqq
				\begin{cases*}
					1  & \(c\) enthält \(a\) in positiver Richtung \\
					-1 & \(c\) enthält \(a\) in negativer Richtung \\
					0  & sonst
				\end{cases*}
		\end{equation*}
		Ein Vektor \( \vec{x} \in \R^n \) ist eine periodische Tension gdw. \( \mat{\Gamma} \vec{x} \equiv_T \vec{0} \) gilt.
		
		Ein Spannbaum \( \mathcal{T} = \mathcal{T}^l + \mathcal{T}^u \) bestimmt ein eindeutiges Potential \( \vec{\pi}^{(\mathcal{T})} = (\pi_i) \), welches jedem Knoten einen Wert \( \pi_i \in \R^n \) zuordnet. Für dieses Potential gilt \( \forall (a : i \to j) \in \mathcal{T}^l : \pi^{(\mathcal{T})}_j - \pi^{(\mathcal{T})}_i = l_a \) und \( \forall (a : i \to j) \in \mathcal{T}^u : \pi^{(\mathcal{T})}_j - \pi^{(\mathcal{T})}_i = u_a \).
		
		% TODO: Stopped here.
	% end


	\bibliography{cite}
\end{document}



































