\documentclass[a4paper, 11pt, twocolumn]{scrartcl}

% Generale page style.
\usepackage[margin = 1cm]{geometry}

% Core packages.
\usepackage[ngerman]{babel}
\usepackage[utf8]{inputenc}
\usepackage[T1]{fontenc}
% Other packages.
\usepackage{amssymb}
\usepackage{csquotes}
\usepackage{enumitem}
\usepackage[mathcal]{euscript}  % Get readable mathcal font.
\usepackage{float}
\usepackage{listings}
\usepackage{makecell}
\usepackage{mathtools}
\usepackage{newunicodechar}
\usepackage{stmaryrd}
\usepackage{tabto}
\usepackage[disable]{todonotes}
%\usepackage{todonotes}

% Basic information.
\title{Lösung periodischer Fahrplanoptimierungsprobleme durch Modulo-Simplex-Berechnungen}
\author{Fabian Damken}
\date{\today}

% Description-list styling.
\SetLabelAlign{parright}{\parbox[t]{\labelwidth}{\raggedleft#1}}
\setlist[description]{style = multiline, leftmargin = 4cm, align = parright}

\MakeOuterQuote{"}

% New commands.
\newcommand{\const}{\ensuremath{\textrm{const}}}
\newcommand{\C}{\ensuremath{\mathbb{C}}}
\newcommand{\dif}[1]{\ensuremath{\,\mathrm{d}#1}}
\newcommand{\N}{\ensuremath{\mathbb{N}}}
\newcommand{\qed}{\hfill \(\Box\)}
\newcommand{\R}{\ensuremath{\mathbb{R}}}
\newcommand{\Z}{\ensuremath{\mathbb{Z}}}
\renewcommand{\vec}[1]{\mathbf{#1}}
\makeatletter
\newcommand{\BIG}{\bBigg@{4}}
\newcommand{\BIGG}{\bBigg@{5}}
\makeatother
\DeclareMathOperator{\arr}{arr}
\DeclareMathOperator{\dep}{dep}

\newcommand{\subsubparagraph}[1]{\hspace{1cm} \textbf{#1:}}

\newcommand{\zB}{z.\,B.~}

\newunicodechar{−}{--}

\begin{document}
	\bibliographystyle{alpha}

	\makeatletter
	\begin{center}
		\textbf{\Large Lösung periodischer \mbox{Fahrplanoptimierungsprobleme} \\ \vspace{2mm} durch Modulo-Simplex-Berechnungen} \\ \vspace{3mm}
		\textbf{\large Fabian Damken} \\ \vspace{1mm}
		{\large \@date}
	\end{center}
	\makeatother

	\section{Einleitung}
		In den letzten Jahren sind periodische Fahrpläne und die Optimierung dieser weit in den Fokus der Optimierungsforschung gerückt. Die meisten Ergebnisse bauen auf den Modellen \cite{serafiniMathematicalModelPeriodic1989} auf. Diese Modelle erlauben eine flexible Modellierung von Taktfahrplänen und vielen Voraussetzungen (\zB Vorfahrtbeschränkungen und Grenzwerte für gleichzeitig fahrende Züge).
		
		Die folgenden Abschnitte beschäftigen sich mit der grundlegenden Modellierung solcher Schienennetze.
		
		Ein \emph{Schienennetz} ist ein System an Zuglinien \(\mathcal{L}\) und Stationen \(\mathcal{S}\). Jede Linie \(L \in \mathcal{L}\) ist dabei als eine Transportkette zu Verstehen, deren Züge eine bestimmte Sequenz an Stationen bedienen. Wenn eine Linie \(L \in \mathcal{L}\) eine Station \(S \in \mathcal{S}\) bedient, dann sind \( (L, \arr, S) \) und \( (L, \dep, S) \) die Ankunfts- und Abfahrt-Ereignisse ("Events") von \(L\) an \(S\). Ein Zeitplan weißt jedem Ereignis \( i = (L, \arr, S) \) (oder \( i = (L, \dep, S) \)) eine Zeit \( \pi_i \in \R \) zu. Eine Aktivität \( a : i \to j \) ist ein Prozess, welcher insgesamt die Zeit \( x_a \coloneqq \pi_j - \pi_i \) benötigt.
		
		Jeder Start- und Stopp-Aktivität wird eine Zeitspanne \( \Delta_a = [l_a, u_a] \) zugeordnet, wobei \( l_a \) die minimale und \( u_a \) die maximale Lauf- und Stoppzeit darstellt. Sie stellen damit eine obere und untere Schranke dar. Der Zeitplan \( \vec{\pi} = (\pi_i) \) heißt \emph{durchführbar}, wenn \( x_a \in \Delta_a \) für alle Aktivitäten \( a : i \to j \) gilt. Abgesehen von Start- und Stoppaktivitäten können viele reale Einschränkungen durch solche Zeitspannen definiert werden (\zB Vorfahrtbeschränkungen zur Betriebssicherheit oder Umsteigezeiten für Passagiere). Insgesamt ergibt sich aus den Aktivitäten \(\mathcal{A}\) und den Stationen \(\mathcal{S}\) ein gerichtetes Netzwerk, welches \emph{Ereignis-Aktivitäts-Netzwerk} genannt wird.
		
		Solche nicht-periodische Probleme sind leicht mit Kürzester-Weg-Berechnungen lösbar. Zur Betrachtung von periodischen Fahrplänen muss das Modell weiter angepasst werden. Ein \emph{periodischer Zeitplan} weißt jedem Ereignis eine Zeit \( \pi_i \in \R \) zu, wobei das Ereignis immer zu den Zeitpunkten \( \pi_i + z_i T \) mit \( z_i \in \Z \), dem \emph{Modulo-Parameter} stattfindet. Aus Gründen der Einfachheit wird angenommen, dass die Zeitperiode \(T\) für das ganze System gleich ist. Um unterschiedliche Perioden zu modellieren, wird \(T\) als das größte gemeinsame Vielfache aller Perioden gewählt.
		
		Die Lösung eines solchen Problems ist gegeben durch einen Vektor \( \vec{\pi} = (\pi_i) \). Mit den unteren und oberen Schranken \( l_i, u_i \in \R \) wird die Durchführbarkeit von nicht-periodischen Lösungen auf periodische Lösungen erweitert. Eine periodische Lösung \( \vec{\pi} \) heißt \emph{durchführbar}, wenn \( \forall a : i \to j : \exists z_a \in \Z : l_a \leq x_a - z_a T \leq u_a \) gilt. Die obere (\( y_a^\text{low} \coloneqq [x_a - l_a]_T \)) unter untere (\( y_a^\text{upp} \coloneqq [u_a - x_a]_T \)) \emph{Pufferzeit} ("slack time") ist die Zeit, um welche die \emph{tension} \(x_a\) erhöht oder verringert werden kann. Der Modulo-Operator ist definiert durch \( [t]_T \coloneqq \min \big\{ t + zT \,\vert\, z + zT \geq 0 \big\} \).
		
		%Da die unteren und oberen Schranken durch Invertieren der Richtung der Kante \(a\) ausgetauscht werden können, kann 5das Optimierungsproblem allein in Bezug auf die untere Pufferzeit \( y_a^\text{low} \) formuliert werden:
		%\begin{equation*}
		%	\min \Bigg\{ \sum_{a : i \to j} \vec{\omega}_a (x_a - l_a - z_a T) \,\big\vert\, \forall a : i \to j : l_a \leq x_a %- z_a T \leq u_a,\, z_a \in \Z \Bigg \}
		%\end{equation*}
	% end


	\bibliography{cite}
\end{document}



































