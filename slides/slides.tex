\documentclass[accentcolor = tud11b, colorbacktitle, landscape, german, presentation]{tudbeamer}

% Core packages.
\usepackage[ngerman]{babel}
\usepackage[utf8]{inputenc}
\usepackage[T1]{fontenc}
% Other packages.
\usepackage{csquotes}
\usepackage{float}
\usepackage{listings}
\usepackage{makecell}
\usepackage{newunicodechar}
\usepackage{stmaryrd}
\usepackage{tabto}
%\usepackage[disable]{todonotes}
\usepackage{todonotes}
% TikZ.
\usepackage{tikz}
\usetikzlibrary{angles, arrows.meta, backgrounds, calc, positioning, shapes}

% Basic information.
\title{Lösung periodischer Fahrplanoptimierungsprobleme durch Modulo-Simplex-Berechnungen}
\institute{Fabian Damken}
\date{\today}

\MakeOuterQuote{"}

\tikzset{> = { Latex[length = 2mm] }}

\AtBeginSection{
	%\begin{frame}
	%	\centering
	%	\vspace{1.5cm}
	%	\Huge \textbf{\insertsectionhead}
	%\end{frame}

	\begin{frame}{\insertsectionhead \\ Gliederung}
		\tableofcontents[currentsection]
	\end{frame}
}

% New commands.
\newcommand{\arr}{\mathit{arr}}
\newcommand{\const}{\ensuremath{\textrm{const}}}
\newcommand{\C}{\ensuremath{\mathbb{C}}}
\newcommand{\dep}{\mathit{dep}}
\newcommand{\dif}[1]{\ensuremath{\,\mathrm{d}#1}}
\newcommand{\N}{\ensuremath{\mathbb{N}}}
\newcommand{\R}{\ensuremath{\mathbb{R}}}
\newcommand{\Z}{\ensuremath{\mathbb{Z}}}
% Matrix and vector notation. Use both boldsymbol and mathbf as the first works
% only for greek letters while the latter works only for latin letters.
\newcommand{\mat}[1]{\boldsymbol{\mathbf{#1}}}
\renewcommand{\vec}[1]{\boldsymbol{\mathbf{#1}}}
\makeatletter
\newcommand{\BIG}{\bBigg@{4}}
\newcommand{\BIGG}{\bBigg@{5}}
\makeatother

\newcommand{\zB}{z.\,B.~}

\newunicodechar{−}{--}

\begin{document}
	\bibliographystyle{alpha}
	
	\begin{titleframe}
		\onslide<2->{\tableofcontents}
	\end{titleframe}

	\section{Einleitung und Motivation}
		\todo{Einleitung und Motivation}
	% end
	
	\section{Mathematische Modellierung}
		\begin{frame}{Bahnnetz und Events}
			Ein \emph{Bahnnetz} ist ein System von
			\begin{itemize}
				\item Linien    \tabto{1.5cm} \(\mathcal{L}\) und
				\item Stationen \tabto{1.5cm} \(\mathcal{S}\).
			\end{itemize}
		
			\vspace{1cm}
			Bedient Linie \( L \in \mathcal{L} \) die Station \( S \in \mathcal{S} \), so sei
			\begin{itemize}
				\item \( (L, \arr, S) \) \tabto{1.6cm} das Ankunfts- und
				\item \( (L, \dep, S) \) \tabto{1.6cm} das Abfahrts-\emph{Event}.
			\end{itemize}
		
			Eine Linie ist eine alternierende Sequenz von Ankunfts- und Abfahrts-Events.
		\end{frame}
		\note[itemize]{
			\item Linie \( L \in \mathcal{L} \) ist Transportkette.
			\item Züge aus \(L\) bedienen bestimmte Sequenzen an Stationen.
		}
	
		\begin{frame}{Fahrpläne und Aktivitäten}
			Ein \emph{Fahrplan} \( \vec{\pi} = (\pi_i) \) weiß jedem Event \( i = (L, \arr, S) \) (bzw. \( i = (L, \dep, S) \)) einen Zeitpunkt \( \pi_i \in \R \) zu.
			
			\vspace{1cm}
			Eine \emph{Aktivität} \( a : i \to j \) beschreibt den Übergang von \(i\) zu \(j\). Die hierfür gebrauchte Zeit ist
			\begin{equation*}
				x_a = \pi_j - \pi_i
			\end{equation*}
			Sei \( \mathcal{A} \) die Menge aller Aktivitäten.
		\end{frame}
		\note[itemize]{
			\item Zeitpunkt \( \pi_i \) ist Zeitpunkt, an dem das Ereignis eintritt.
		}
	
		\begin{frame}{Zeiteinschränkungen}
			Jeder Aktivität \( a \in \mathcal{A} \) wird eine Zeitspanne
			\begin{equation*}
				\Delta_a = [l_a, u_a]
			\end{equation*}
			zugeordnet.
			
			Ein Fahrplan \(\vec{\pi}\) ist \emph{durchführbar} ("feasible"), wenn
			\begin{equation*}
				\forall a \in \mathcal{A} : x_a \in \Delta_a \quad\iff\quad \forall (a : i \to j) \in \mathcal{A} : l_a \leq \pi_j - \pi_i \leq u_a
			\end{equation*}
		\end{frame}
		\note[itemize]{
			\item \(l_a\) untere und \(u_a\) obere Schranke
		}
	
		\begin{frame}{Aussagekraft von Zeiteinschränkungen}
			\begin{itemize}
				\item Fast alle realen Einschränkungen können durch Zeitspannen beschrieben werden.
				\begin{itemize}
					\item<2-> Fahrzeit eines Zuges
					\item<2-> Sicherheitseinschränkungen (\zB Vorfahrt)
					\item<2-> Wartezeiten (Kundenzufriedenheit)
					\item<2-> Umsteigezeiten
					\item<2-> \dots
				\end{itemize}
			\end{itemize}
		\end{frame}
	
		\begin{frame}{Event-Aktivitäts-Netzwerk}
			\vspace{-0.3cm}
			\begin{figure}[H]
				\centering
				\begin{tikzpicture}[event/.style = { draw, rectangle, rounded corners, minimum width = 2cm, minimum height = 0.6cm }]
					\node [event]                                         (L1-dep-B) {\((L_1, \dep, B)\)};
					\node [event, above left = of L1-dep-B, xshift = 1cm] (L1-arr-C) {\((L_1, \arr, C)\)};
					\node [event, below = of L1-dep-B]                    (L1-arr-B) {\((L_1, \arr, B)\)};
					\node [event, below left = of L1-arr-B, xshift = 1cm] (L1-dep-A) {\((L_1, \dep, A)\)};
					
					\node [event, right = of L1-dep-B, xshift = 2cm]        (L2-arr-B) {\((L_2, \arr, B)\)};
					\node [event, above right = of L2-arr-B, xshift = -1cm] (L2-dep-E) {\((L_2, \dep, E)\)};
					\node [event, below = of L2-arr-B]                      (L2-dep-B) {\((L_2, \dep, B)\)};
					\node [event, below right = of L2-dep-B, xshift = -1cm] (L2-arr-D) {\((L_2, \dep, D)\)};
					
					\draw [->] (L1-dep-A) -- node[above left]{run arc} (L1-arr-B);
					\draw [->] (L1-arr-B) -- node[left]{stop arc}      (L1-dep-B);
					\draw [->] (L1-dep-B) -- node[below left]{run arc} (L1-arr-C);
					
					\draw [->] (L2-dep-E) -- node[below right]{run arc} (L2-arr-B);
					\draw [->] (L2-arr-B) -- node[right]{stop arc}      (L2-dep-B);
					\draw [->] (L2-dep-B) -- node[above right]{run arc} (L2-arr-D);
					
					\draw [->] (L2-arr-B) -- node[midway, sloped, above, yshift = -0.1cm]{change arc} (L1-dep-B);
					\draw [->] (L1-arr-B) -- node[midway, sloped, below]{change arc}                  (L2-dep-B);
					
					\draw [->] (L1-dep-B) -- node[midway, sloped, above, yshift = -0.1cm]{headway arc} (L2-dep-B);
				\end{tikzpicture}
				\caption{Event-Aktivitäts-Netzwerk, vgl. \cite[Fig.~1]{nachtigallSolvingPeriodicTimetable2008}}
				\label{fig:event_activity_network}
			\end{figure}
		\end{frame}
		\note[itemize]{
			\item "headway" = Vorfahrt
		}
	
		% TODO: Stopped here.
	% end
	
	\section{Modulo-Simplex-Berechnungen}
		\todo{Modulo-Simplex-Berechnungen}
	
		\subsection{Grundlagen von Modulo-Simplex}
			\todo{Grundlagen von Modulo-Simplex}
		% end
		
		\subsection{Periodisches Modulo-Simplex}
			\todo{Periodisches Modulo-Simplex}
		% end
	% end
	
	\section{Zusammenfassung}
		\todo{Zusammenfassung}
	% end

	\begin{frame}{Literatur}
		\bibliography{../cite}
	\end{frame}
\end{document}
